% Chapter Template

\chapter{Introduction} % Main chapter title

\label{Introduction} % Change X to a consecutive number; for referencing this chapter elsewhere, use \ref{ChapterX}

\section{Introduction}
In aceasta lucreare, voi prezenta demersul si technologiile folosite pentru un proiect iot in domeniul healthcare.

\section{Motivation}
From my point of view software development is about humanity, about helping people using technology.

In 5 years of studding and 4 years beaning a Software Developer in different companies and fields I learned how to build software for deferent proposes marketing, sales and automotive.

In cei 5 ani de unversitate si 4 ani de lucru in cadrul a diferite companii ca software developer in diferite domenii am invatat cum technologia poate ajuta omenirea in tot felul de moduri.

Eu consider ca a fi un software developer iti ofera posibilitati nelimitate de a te ajuta pe tine si pe cei din jur.
Medicina este unul dintre cele mai interesante domenii prin care poti sa ajuti omenirea, clar un sw dev nu poate salva vieti cum poate un doctor dar un sw dev poate dezvolta unelte care pot ajuta la salvarea vietilor.
Asa ca am decis ca este momentul sa dezvolt ceva util pentru populatie.
Proiectul de fata IoT Healthcare este o aplicatie  care poate ajuta entitati precum spitale sau azile de batrani la o analizare mult mai detaliata a pacientilor. Eu cred ca daca voi reusi sa introduc acest proiect intr-un spital chiar va fi de mare ajutor.


\section{Target}

Iot healthcare pune la dispozitie unor personane de specialitate o platforma de analiza a pacientilor.
Aplicatia permite conecatarea a diferitor senzori la o aplicatie backend care poate rula analize pe aceste date.
Momentan aplicatia permite conecatarea cu un smartband care poate inregistra activitati precum numarul de pasi, bataile inimii, durata somnului, toate aceste date sunt salvate in baza de date si pot fi acesate din interfata web. Avand aceste informatii putem extrage statistici gen corelare dintre varsta si ritm cardiac, daca un tip de medicament te face sa dormi mai mult sau monitorizarea activitatii fizice a pacientului in functie de evolutia afectiunii etc.