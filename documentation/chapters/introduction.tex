% Chapter Template

\chapter{Introduction} % Main chapter title

\label{Introduction} % Change X to a consecutive number; for referencing this chapter elsewhere, use \ref{ChapterX}

\section{Introduction}
Internet of things (IoT) devices and services will guide healthcare toward the new generation of proficient services while saving time and lives with high accuracy. Continuous progress in remote healthcare domain leads to the invention of wireless devices. Nowadays, a comprehensive view of the patient’s overall health can be obtained through prototypes of the next generation emergency rooms. IoT technology facilitates the intercommunication and can alert the hospital staff based on the patient’s vitals. It transforms the healthcare industry to more efficient, low costand better patient care systems generation. Concurrently, the current era is filled with milestone improvements in the healthcare technology that has resulted in the bigdata world. Big data is characterized by volume, variety, velocity and veracity. Bigdata and IoT are topical emerging technologies that attract focus from severalresearchers and engineers to develop the next generation of smarter connected devices/products. In addition, embedded medical devices are spreading to provides accessible information anywhere in the world. Healthcare organizations use the continuous engineering potential to design powerful future. They are intimately connected with enormous Internet-connected things/devices that generate massive amounts of data. Superior healthcare outcome is the foremost objective shared by hospitals, clinics and health organizations around the world. Healthcare combined with IoT drives the innovation of medical services along with cost reduction, accuracy improvement, and wide spread of the healthcare services to more population. Recently, healthcare stakeholders directed their attention toward big data analyses to support the healthcare sector instead of developing only automated medical systems, and digitizing medical records. Big data has the potential to assist healthcare organizations to equip providers with the necessary tools for better healthcare. The IoT healthcare devices acquire medical information in the form of signals and/or images. These acquired big data from the IoT requires analysis for accurate diagnosis. This provides the healthcare organizations with the require detailed data to achieve effective population health management. An unprecedentedlevel of real-time data can be obtained by intelligent systems. Meanwhile, big data runs on open-source technologies with inconsistent security schemes. Thus, healthcare organizations have to ensure superior big data security. (Chintan Bhatt, Nilanjan Dey,Amira S. Ashour 2017)
\newline

The Internet of Things (IoT) is a physical device along with other item network that embedded with software, electronics, network connectivity, and sensor to collect objects in order to exchange data. The IoT impact in healthcare is still in its initial development phases. The IoT system has several layers that lead to implementation challenges where many engaged devices have sensors to collects data. Each has its manufacturer own exclusive protocols. These protocols using soft ware environment associated with privacy and security raise new challenges in the IoT technology. This current chapter attempts to understand and review the IoTconcept and healthcare applications to realize superior healthcare with affordable cost. The chapter included in brief the IoT functionality and its association with the sensing and wireless techniques to implement the required healthcare applications.(C. Bhatt, 2017)
\newline

The next era will be the connection between the physical things and internet. The things include goods, machine, appliances even we also become the part of it. The reason for integrating healthcare with Internet of Things features into medical devices improves the quality and effectiveness of service, bringing especially high value for elderly, patients with chronic conditions and those that require consistent supervision. Now research is going on-how to transform healthcare industry by increasing efficiency, lowering costs and put the focus back on better patient care. The Internet of Things will be a game-changer for the healthcare industry. With an intelligent system and powerful algorithms, one can obtain unprecedented level of real-time, life-critical data which is captured and analyzed to drive people to advance research, management, and critical care. Taking care of patient's health at very low cost is an important factor. The main idea of applyingIoT in healthcare is to move out from traditional area to visit hospitals and this waiting will come to an end. The concept here is that it can be able to sense, process and communicate with biomedical and physical parameters so that they can work on it. Many applications and devices have been designed for healthcare purpose and have been put for people to use. The view is to connect the doctors, patients, and nurses via smart device and each entity can roam without any restrictions. The idea is 24 * 7 monitoring of patient.(Yesha Bhatt, Chintan Bhatt, 2017)
\newline

%In this paper I will present the steps and technologies used for an IOT project in the healthcare field.

\section{Motivation}
From my point of view, software development is a means of helping people in various fields and making their lives easier with technology.
In 5 years of studying  and 4 years working as a Software Developer in different
companies and fields I learned how to create software for different proposes, like marketing,
sales and automotive.

I strongly believe that being a software developer gives you endless possibilities to help yourself and those around you. The medical field offers some of the most exciting possibilities to develop innovative technology that can make an actual difference in people's lives. Obviously, a Software Developer cannot replace an actual doctor, but they can develop tools to aid the said doctor in saving lives. With this idea in mind, I decided that the best use of my knowledge is in the medical field and I wanted to develop an application that would be useful to the population.
Thus, IOT Healthcare is an application that can be used in hospitals and other medical facilities, like asylums, hospices or nursing homes to better analyze the condition of the patients. I am convinced that if my project will eventually be implemented in these facilities, it will be a great help for the caretakers and patients alike.



\section{Target}

IOT Healthcare provides an analysis platform for specialists, where they can monitor the patient's symptoms. The system allows the concatenation of different sensors to a back-end application that can run analysis on the provided data. For now, the application can be connected to a smart band that can register activities like number of steps, heart beat, sleep duration. All this data is saved in the database and can be accessed from the web interface. With this information made available to us, we can extract statistics like the correlation between age and heart rate, if a certain medicine makes you sleep more or monitoring the patient's physical activity according to the evolution of their affection, etc.