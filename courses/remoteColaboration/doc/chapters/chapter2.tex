% Chapter Template

\chapter{Description of online meetings} % Main chapter title

\label{Meetings} % Change X to a consecutive number; for referencing this chapter elsewhere, use \ref{ChapterX}

\section{Meeting 1}
\begin{center}
\begin{tabular}{| c | c | c }
	\hline
	Date &  31.03.2017   \\
	\hline
	Location & Skype  \\
	\hline
	Atendees & All the team members   \\
	\hline
	Duration & 1h  \\
	\hline
\end{tabular}
\end{center}
We settled on Wiggio to take the meetings on Skype because it features good chat functions and video calls, screen sharing. 
We had the following agendas for the meeting: 
\begin{itemize}
	\item	Introductory round
	\item	Facilitator and minute taker roles
	\item	Discuss Meeting schedule
\end{itemize}

So when we successfully connect to the meeting the very first thing we did, was getting know with each other in an introductory round. This is very important in a team where the members do not know each other, in order to make every member feel comfortable. 

The second thing was to select slack for communication, we select slack because is dedicated for this kind of projects and teams. I set up a team on slack: \href{https://remc17white.slack.com}{remc17white.slack.com}, I did it because I have it experience with slack in the past.

We have made the decision to do a rotation of the roles for each meeting.

We talked between each other about what we have to do, in order to be sure that everybody understood the course requirements.
 
\section{Meeting 2}
\begin{center}
	\begin{tabular}{| c | c | c }
		\hline
		Date &  04.04.2017\\
		\hline
		Location & Skype  \\
		\hline
		Atendees & All the team members   \\
		\hline
		Duration & 1h  \\
		\hline
	\end{tabular}
\end{center}

The colleagues from Germany have found an implementation of code for collecting the tweets they also  described
in a presentation. I also contributed with a tool for scanning the facebook profile.

\textbf{New theme for the project}
\begin{itemize}
	\item We discussed (promoted by me), in the following,
	the possibility to choose other theme for the project which
	only Twitter would be necessary.
	\item 	It was mentioned a current topic which a game called
	Blue Whale was influencing people to suicide
	\item One of the suggestions was to use Twitter to find
	possible suicides (not necessarily influenced by the game)
\end{itemize}
\textbf{Agreement:}
\begin{itemize}
	\item The conclusion about the suggestion was it could be too
	complicated to label these messages
	\item Requires knowledge and research in topics we don't
	know (e.g. Psychology)
\end{itemize}


\section{Meeting 3}
\begin{center}
	\begin{tabular}{| c | c | c }
		\hline
		Date & 	07.04.2017   \\
		\hline
		Location & Skype  \\
		\hline
		Atendees & All the team members   \\
		\hline
		Duration & 1h  \\
		\hline
	\end{tabular}
\end{center}
In the beginning of the third meeting we explored the idea of a new approach for retrieving and analyzing the data Paul suggested that we could also use another web application, which is called "sentiment vz" insted of keyhole.
The problem was that this application is solely
retrieving data from Twitter, and this is a problem to
accomplish our project goal, which is to compare the
sentimental analysis from two different social
platforms
For the topic, which we wanted to use for collecting the
data, we were discussing about current events in the Brexit or the happenings in Syria.
Another topic we were discussing was the French
presidential election. A possible problem of this
problem could be that most of the data might be in
French, which make it difficult to analyze, since most of
the sentimental analyses work with the English
language.

\section{Meeting 4}
\begin{center}
	\begin{tabular}{| c | c | c }
		\hline
		Date & 	11.04.2017   \\
		\hline
		Location & Skype  \\
		\hline
		Atendees & All the team members   \\
		\hline
		Duration & 1h  \\
		\hline
	\end{tabular}
\end{center}
Bernardo had started the conversation and asked from every
participant about the research progress which everyone have
from previous meeting.

All members shared their ideas what they have done so far.
Paul shared the key words and hashtags for extracting data
from both social medias.

\section{Meeting 5}
\begin{center}
	\begin{tabular}{| c | c | c }
		\hline
		Date & 	25.04.2017   \\
		\hline
		Location & Skype  \\
		\hline
		Atendees & All the team members   \\
		\hline
		Duration & 1h  \\
		\hline
	\end{tabular}
\end{center}


Bernardo started the discussion to find out the results of the
data which we have gotten since last meeting. Benjamin
shared his approach for getting the data from YouTube
comments about "Brexit" and shared his experienced about
some useless comments due to useless content in the video.
Diana explained that Paul get the data from twitter and
Bogdan was responsible for sentiment analysis with that data.

\section{Meeting 6}
\begin{center}
	\begin{tabular}{| c | c | c }
		\hline
		Date & 	28.04.2017   \\
		\hline
		Location & Skype  \\
		\hline
		Atendees & All the team members   \\
		\hline
		Duration & 1h  \\
		\hline
	\end{tabular}
\end{center}
 \textbf{Organisation}
 \begin{itemize}
	\item talked about how the graphical data should look on the final paper;
	\item talked about gender specific data, on how we could include this parameter into our research paper;
	\item the Youtube team talked about gender specific data, ex. It turned out that women tend to be less negative/hateful;
 \end{itemize}


\section{Meeting 7}
\begin{center}
	\begin{tabular}{| c | c | c }
		\hline
		Date &	02.05.2017   \\
		\hline
		Location & Skype  \\
		\hline
		Atendees & All the team members   \\
		\hline
		Duration & 1h  \\
		\hline
	\end{tabular}
\end{center}


Unfortunately, I was not able to participate in this meeting. I also announced my colleges before I had an important meeting at work.

\section{Meeting 8}
\begin{center}
	\begin{tabular}{| c | c | c }
		\hline
		Date &  05.05.2017   \\
		\hline
		Location & Skype  \\
		\hline
		Atendees & All the team members   \\
		\hline
		Duration & 1h  \\
		\hline
	\end{tabular}
\end{center}

Reviewing the chapter “Introduction”

We discussed about the content of the introduction chapter
of our research paper. It was suggested to add the purpose
of the research paper to our introduction because in this part
of a paper it is necessary to mention the overall goal.

It was asked if information like the importance of social
media in terms of the world of work, should be part of the
introduction or not. The reasons for these doubts were
because it did not seem relevant to our topic.

Another suggestion, which was discussed, was the
comparison between the social platforms itself like platforms,
which focus more on file sharing, or focusing more about the
communication.

Furthermore, it was said that mentioning the behavior in
social platforms comparing with interaction from the real
world could also be part of the introduction. This may also
be linked to the degree of exposure of private information.
\section{Meeting 9}
\begin{center}
	\begin{tabular}{| c | c | c }
		\hline
		Date & 	09.05.2017   \\
		\hline
		Location & Skype  \\
		\hline
		Atendees & All the team members   \\
		\hline
		Duration & 1h  \\
		\hline
	\end{tabular}
\end{center}


In the last meeting we have discussed the research paper by proof reading each part of the paper to remove grammatical mistakes. Every group member has studied the individual part of paper
and gave some suggestions to replace certain phrases. I provided the twitter result with both graphically and tabular form after cleaning the data from tweets and
added in the result of the research paper

